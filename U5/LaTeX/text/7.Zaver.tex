\section{Závěr}

Byl vyhotoven skript v MATLABu a pythonu, který generuje dvě dvourozměrné datové sady a provádí transformaci pomocí hlavních komponent (PCA). První datová sada má podlouhlý charakter a je natočená, zatímco druhá datová sada má charakter kružnice. Výsledky ukazují, že v první datové sadě obsahuje první hlavní komponenta alespoň 70\% informace, zatímco ve druhé datové sadě je rozdíl v informaci mezi hlavními komponentami menší než 10\%.

\subsection{Možné oblasti pro vylepšení}

\begin{itemize}
    \item \textbf{Zadání uživatelských vstupů}: Uživatel by mohl zadávat parametry jako uživatelský vstup.
    \item \textbf{n-dimenzionální data}: Kód by mohl být rozšířen tak, aby fungoval s n-dimenzionálními daty.
\end{itemize}