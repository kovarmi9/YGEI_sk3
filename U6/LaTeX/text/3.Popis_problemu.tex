\section{Popis problému}
Archeologická lokalita Klobásná u Veselí nad Lužnicí byla zdokumentována pomocí technologie laserového skenování, která poskytla detailní bodové mračno reprezentující povrch terénu. Tato data umožňují přesné zachycení topografických vlastností zkoumané oblasti, avšak jejich surová podoba zahrnuje nejen reliéfní prvky, ale také vegetaci, nadzemní objekty a šum.
\\
Vegetace a nadzemní objekty, jako jsou stromy, keře či stavby, mohou zakrývat archeologicky významné struktury, například mohyly, a tím ztěžovat jejich identifikaci. Kromě toho mohou být data zatížena šumem, což jsou nežádoucí body vzniklé například odrazy laserového paprsku od malých částic, chybami měření nebo nepřesnostmi při sběru dat. Tento šum může dále komplikovat analýzu a zvyšovat potřebu datového čištění.
