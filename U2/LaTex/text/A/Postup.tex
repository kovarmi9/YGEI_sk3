\subsection*{Část A}

\begin{itemize}
    \item \textbf{Výběr vzoru}: Výběr konkrétního vzorku, který bude použit jako referenční obrazec pro vyhledávání v obraze. Tento vzor může být vybrán interaktivně uživatelem nebo může být předdefinován. K výběru vzorků slouží funkce \texttt{select\_samples.m}. Pro přepnutí mezi interaktivním výběrem vzorku a využitím předdefinovaného vzorku slouží část kódu: 
    \begin{verbatim}
    interactive_selection = 'YES';
    \end{verbatim}
    
    \item \textbf{Průměrný vzorek}: Pokud je vybráno více vzorků, bude k vyhledávání použit vzorek získaný jako průměr všech vybraných vzorků.V Matlabu je průměrný vzorek vypočten jako:
    \begin{verbatim}
    template = zeros(template_size, 'double');
    for i = 1:num_samples
        template = template + double(templates{i});
    end
    template = uint8(template / num_samples);
    \end{verbatim}
    
    Matematicky lze průměrný vzorek \(T_{\text{avg}}\) vypočítat jako:
    \[
    T_{\text{avg}}(x, y) = \frac{1}{N} \sum_{k=1}^{N} T_k(x, y)
    \]
    kde \(T_k(x, y)\) je hodnota pixelu na pozici \((x, y)\) v \(k\)-tém vzorku a \(N\) je celkový počet vzorků. Tento průměrný vzorek pak slouží jako referenční obrazec pro vyhledávání v obraze.
    
    \item \textbf{Výběr vhodného barevného kanálu}: Po výběru vzoru pro vyhledávání v obraze je vhodné převést obraz i vzorek do barevného prostoru, kde jsou zájmové objekty dobře rozpoznatelné. Například převod do barevného prostoru YCbCr a použití kanálu Y může zvýraznit jasové informace. Matematicky lze převod z RGB do YCbCr vyjádřit jako:
    \[
    \begin{pmatrix}
    Y \\
    C_b \\
    C_r
    \end{pmatrix}
    =
    \begin{pmatrix}
    0.2990 & 0.5870 & 0.1140 \\
    -0.1687 & -0.3313 & 0.5000 \\
    0.5000 & -0.4187 & -0.0813
    \end{pmatrix}
    \begin{pmatrix}
    R \\
    G \\
    B
    \end{pmatrix}
    +
    \begin{pmatrix}
    0 \\
    128 \\
    128
    \end{pmatrix}
    \]
    kde \(Y\) představuje jasovou složku, \(C_b\) a \(C_r\) jsou chrominační složky.\cite{pIDhmNtdwMgbcGoe}.

    \item \textbf{Úprava obrazu pomocí pohyblivého filtru (kernelu)}: Obraz je možné dále upravit například pomocí kernelu. Pohyblivé filtry lze rozdělit na vysokofrekvenční a nízkofrekvenční. Vysokofrekvenční filtry zesilují rozdíly mezi hodnotami v obraze a nízkofrekvenční filtry potlačují rozdíly mezi hodnotami v obraze, což vede k vyhlazení obrazu.\cite{DPZFiltrace} Gaussův filtr je nízkofrekvenční filtr, pro jehož aplikaci lze využít vestavěnou funkci v MATLABu \texttt{imgaussfilt}.\cite{mathworksGauss} Tato funkce přijímá povinný parametr \texttt{input\_image}, který představuje rastr, a volitelný parametr, který značí směrodatnou odchylku. Použití funkce v MATLABu je následující:
    \[
    \texttt{output\_image = imgaussfilt(input\_image, 2);}
    \]
    Matematicky lze Gaussův filtr popsat jako metodu, při které se pro jednotlivé pixely pokryté kernelem vypočtou váhové koeficienty podle Gaussovy funkce:
    \[
    G(i,j) = \exp\left(-\frac{(i-u)^2 + (j-v)^2}{2\sigma^2}\right)
    \]
    kde \(i, j\) jsou libovolné pixely pod kernelem, \(u, v\) jsou střední pixely a \(\sigma\) je směrodatná odchylka, která se zadává jako parametr. Váhy \(W(i, j)\) se určí normalizací hodnot \(G(i, j)\) tak, aby jejich součet byl roven 1:
    \[
    W(i, j) = \frac{G(i, j)}{\sum_{i=-k}^{k} \sum_{j=-k}^{j} G(i, j)}
    \]
    kde \(k\) je poloměr kernelu. Výsledná hodnota (stupeň šedi) pixelu se vypočítá jako vážený průměr hodnot pixelů v okně:
    \[
    P(u,v) = \sum_{i=-k}^{k} \sum_{j=-k}^{j} W(i,j) \cdot V(i,j)
    \]
    kde \(V(i, j)\) je původní digitální hodnota v pixelu \(i, j\) \cite{ZODHGauss}.
    
    Ukázka kernelu pro Gaussův filtr s \(\sigma = 1\) a velikostí 3x3:
    \[
    \renewcommand{\arraystretch}{3}
    \setlength{\tabcolsep}{10pt}
    \begin{array}{|c|c|c|}
        \hline
        0.0751 & 0.1238 & 0.0751 \\
        \hline
        0.1238 & 0.2042 & 0.1238 \\
        \hline
        0.0751 & 0.1238 & 0.0751 \\
        \hline
    \end{array}
    \]

    \item \textbf{Výpočet korelačního koeficientu mezi vzorem a částí obrazu}: Pro výpočet korelace byla využita vestavěná funkce MATLABu \texttt{normxcorr2}, která se používá následujícím způsobem:
    \[
    c = \texttt{normxcorr2(template\_sel, im\_sel)}
    \]
    kde \texttt{template\_sel} je vybraný vzor a \texttt{im\_sel} je část obrazu. Korelační koeficient se vypočítá jako:
    \[
    r = \frac{\sum_{i=1}^{n} (A_i - \bar{A})(B_i - \bar{B})}{\sqrt{\sum_{i=1}^{n} (A_i - \bar{A})^2 \sum_{i=1}^{n} (B_i - \bar{B})^2}}
    \]
    kde \(A_i\) a \(B_i\) jsou hodnoty pixelů ve vzoru a v části obrazu, a \(\bar{A}\) a \(\bar{B}\) jsou průměrné hodnoty těchto pixelů.

    \item \textbf{Vyhledání všech pozic hledaných objektů v obraze}: Prohledání celého obrazu a identifikace všech míst, kde korelační koeficient překračuje stanovenou prahovou hodnotu.

    \item \textbf{Odstranění všech vícenásobných opakování}: Odstranění všech nalezených korelací ve zvoleném rádiusu od místa s největší korelací. k tomu slouží funkce \texttt{find\_unique\_values.m} do níž vstupuje matice hodnot korelací a jejich pozic seřazená vzestupně podle hodnot korelace.
    \begin{verbatim}
    positions = sortrows([vals, rows, cols], -1);
    unique_positions = find_unique_positions(positions, radius);
    \end{verbatim}

    Matematicky lze algoritmus pro hledání unikátních pozic popsat takto:
    
    Vstupem je matice \(\mathbf{P}\), kde každý řádek obsahuje hodnotu korelace a souřadnice \((x, y)\). Matice \(\mathbf{P}\) je seřazena sestupně podle hodnot korelace v prvním sloupci. Rádius \(r\) je minimální vzdálenost, kterou musí mít dvě unikátní pozice od sebe, aby byly považovány za odlišné.
    
    \begin{itemize}
        \item \textbf{Inicializace}: První řádek matice \(\mathbf{P}\) je přidán do matice unikátních pozic \(\mathbf{U}\).
        \item \textbf{Iterace}: Pro každou další pozici \(\mathbf{p}_i\) v matici \(\mathbf{P}\):
        \begin{itemize}
            \item Výpočet vzdálenosti \(d(\mathbf{p}_i, \mathbf{u}_j)\) pro všechny \(\mathbf{u}_j\) v matici \(\mathbf{U}\).
            \item Pokud \(\min_j d(\mathbf{p}_i, \mathbf{u}_j) > r\), je přidáno \(\mathbf{p}_i\) do matice \(\mathbf{U}\).
        \end{itemize}
    \end{itemize}
    
    Vzdálenost mezi dvěma pozicemi \(\mathbf{p}_i = (x_i, y_i)\) a \(\mathbf{p}_j = (x_j, y_j)\) je dána vztahem:
    \[
    d(\mathbf{p}_i, \mathbf{p}_j) = \sqrt{(x_i - x_j)^2 + (y_i - y_j)^2}
    \]
    
    \item \textbf{Výpis a vykreslení nalezených souřadnic}: na závěr jsou nalezené obce s kostelem vykresleny pomocí funkce \texttt{plot} a \texttt{rectangle} a vypsány pomocí funkce \texttt{disp}.

\end{itemize}