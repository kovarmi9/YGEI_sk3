\subsection*{Část A}

Cílem první části je implementovat algoritmus pro vyhledávání vzorů v Müllerově mapě Čech na základě obrazové korelace. Obrazová korelace je metoda, která umožňuje nalezení specifických vzorů v obraze na základě podobnosti mezi vzorem a částí obrazu. Tento úkol zahrnuje následující kroky:

\begin{itemize}
    \item \textbf{Výběr vzoru}: Identifikace a výběr specifického vzoru, který bude použit jako referenční obrazec pro vyhledávání v obraze. V případě že je vybráno více vzorů je vzor pro vyhledávání určen jako průměr všech vybraných vzorů.
    \item \textbf{Výběr vhodného barevného kanálu}: Po výběru vzoru je vhodné převést obraz i vzor do barevného prostoru, kde jsou zájmové objekty dobře rozpoznatelné. Například převod do barevného prostoru YCbCr a použití kanálu Y může zvýraznit jasové informace.
    \item \textbf{Úprava obrazu pomocí pohyblivého fitru (kernelu)}: Obraz je možné dále upravit například pomocí kernelu, například aplikací Gaussova filtru nebo jiných filtrů pro zvýraznění specifických rysů hledaného objektu.
    \item \textbf{Výpočet korelačního koeficientu mezi vzorem a částí obrazu}: Použití matematických metod pro výpočet míry podobnosti mezi vybraným vzorem a různými částmi obrazu.
    \item \textbf{Určení limitní hodnoty korelace pro detekci specifických objektů}: Stanovení minimální hodnoty korelačního koeficientu, která bude považována za dostatečnou pro identifikaci specifického objektu. Tato limitní hodnota se určí experimentálně.
    \item \textbf{Vyhledání všech pozic specifických objektů v obraze}: Prohledání celého obrazu a identifikace všech míst, kde korelační koeficient překračuje stanovenou prahovou hodnotu. Pokud je nalezen jeden objekt vícenásobně může být za nález považován pouze jeden nález tohoto objektu a to ten, který má největší korelaci.
    \item \textbf{Zaznamenání výsledku jako pixelové souřadnice objektů}: Výpis a vykreslení pixelových souřadnic všech nalezených objektů v obraze.
\end{itemize}