
\subsection*{Část A}
Zpracování probíhalo v software MATLAB.


\subsection*{Zobrazení průměrného vzorku}
Průměrný vzorek je zobrazen:
\begin{verbatim}
figure(1)
imshow(template)
title('Average template');
\end{verbatim}

\subsection*{Převod obrázku a vzorku do vybraného kanálu a aplikace kernelu}
Obrázek a vzorek jsou převedeny do vybraného kanálu a je aplikován kernel:
\begin{verbatim}
im_sel = process_image(im, channel, kernel);
template_sel = process_image(template, channel, kernel);
\end{verbatim}

\subsection*{Zobrazení převedeného obrázku}
Převedený obrázek je zobrazen:
\begin{verbatim}
figure(2)
subplot(2,2,2)
imshow(im_sel)
title('Converted picture');
\end{verbatim}

\subsection*{Výpočet korelace}
Korelace mezi vzorkem a obrázkem je vypočítána:
\begin{verbatim}
c = normxcorr2(template_sel, im_sel);
\end{verbatim}

\subsection*{Zobrazení korelace}
Korelace je zobrazena:
\begin{verbatim}
subplot(2,2,3)
imshow(c)
title('Correlation');
\end{verbatim}

\subsection*{Vyhledání pozic s vysokou korelací}
Pozice s korelací nad stanoveným limitem jsou nalezeny:
\begin{verbatim}
[rows, cols] = find(c >= limit);
\end{verbatim}

\subsection*{Kontrola nalezených pozic}
Pokud nejsou nalezeny žádné odpovídající oblasti, je vyvolána chyba:
\begin{verbatim}
if isempty(rows)
    error('No matching areas on the map that correspond for correlation coefficient.');
end
\end{verbatim}

\subsection*{Extrahování hodnot korelace}
Hodnoty korelace na nalezených pozicích jsou extrahovány:
\begin{verbatim}
vals = c(sub2ind(size(c), rows, cols));
\end{verbatim}

\subsection*{Seřazení pozic podle hodnot korelace}
Pozice jsou seřazeny podle hodnot korelace v sestupném pořadí:
\begin{verbatim}
positions = sortrows([vals, rows, cols], -1);
\end{verbatim}

\subsection*{Odstranění vícenásobných opakování}
Vícenásobné opakování jsou odstraněny pomocí funkce \texttt{find\_unique\_positions}:
\begin{verbatim}
unique_positions = find_unique_positions(positions, radius);
\end{verbatim}

\subsection*{Zobrazení nalezených oblastí}
Nalezené oblasti jsou zobrazeny v původním obrázku:
\begin{verbatim}
subplot(2,2,4)
imshow(im)
hold on
for k = 1:size(unique_positions, 1)
    rectangle('Position', [unique_positions(k, 3) - size(template_sel, 2), unique_positions(k, 2) - size(template_sel, 1), size(template_sel, 2), size(template_sel, 1)], 'EdgeColor', 'r')
end
title(['Matching areas: ', num2str(size(unique_positions, 1))]);
hold off
\end{verbatim}

\subsection*{Zobrazení nalezených oblastí v novém obrázku}
Nalezené oblasti jsou zobrazeny v novém obrázku:
\begin{verbatim}
figure(3)
imshow(im)
hold on
for k = 1:size(unique_positions, 1)
    rectangle('Position', [unique_positions(k, 3) - size(template_sel, 2), unique_positions(k, 2) - size(template_sel, 1), size(template_sel, 2), size(template_sel, 1)], 'EdgeColor', 'r')
end
title(['Matching areas: ', num2str(size(unique_positions, 1))]);
hold off
\end{verbatim}

\subsection*{Extrahování pixelových souřadnic}
Pixelové souřadnice nalezených oblastí jsou extrahovány:
\begin{verbatim}
pixel_coordinates = unique_positions(:, 2:3) - size(template_sel) / 2;
\end{verbatim}

\subsection*{Zobrazení pixelových souřadnic}
Pixelové souřadnice jsou zobrazeny:
\begin{verbatim}
disp('Center pixels of matching areas:');
disp(pixel_coordinates);
\end{verbatim}
