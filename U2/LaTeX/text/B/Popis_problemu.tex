\subsection*{Část B}
Cílem je z Topografické mapy ČSSR v měřítku 1 : 25 000 segmentovat plochy lesa pomocí metody $k-Means$. Původní mapa je z roku 1954 a je na ní vidět degradace barev. Mapa byla následně naskenována a komprimována do ztrátového formátu \texttt{*.jpg}. Jelikož se jedná o topografickou mapu, tak plochy lesa obsahují i popisy, vrstevnice cesty a další prvky které segmentaci dělají náročnější.\\
Při zpracování, bylo odhaleno, že zadaná mapa, má jednu část mapy zelenější, což vytváří barevný předěl. Tento úkol zahrnuje následující kroky: 
\begin{itemize}
    \item \textbf{Editace obrazových dat:} změna, kontrastu, jasu a barvených složek pro zvýraznění jednotlivých různých ploch a minimalizace vlivu zeleného předělu.
    \item \textbf{Převedení barevných systémů:} systém $RGB$ převeden do systému $CIE Lab$
    \item \textbf{Segmentace:} obrazová segmentace pomocí vestavěné funkce $k-Means$.
    \item \textbf{Filtrace obrazu:} nově vytvořená vrstva lesa byla filtrována pomocí vyhlazovacího filtru o tvaru disku velikosti $[5x5]$.
    \item \textbf{Mapová algebra:} pomocí mapové algebry byla vypočítána nová vrstva z jednotlivých segmentovaných prvků.
    \item \textbf{Filtrace obrazu:} nově vytvořená vrstva lesa byla filtrována pomocí vyhlazovacího filtru o tvaru disku velikosti $[5x5]$.
    \item \textbf{Export:} export plochy segmentovaného lesa.
\end{itemize}