\section*{Popis problému}
Rastrová data jsou využívána velmi často protože na zobrazení dat s velké většiny nepotřebují specializovaný software, problém nastává u operacích spojených s uschováním, přenosem či editací. Tento problém se projevuje zejména u barevných rastrů s vysokým prostorovým rozlišením a barevnou hloubkou. Pomocí komprimačních algoritmů můžeme zefektivnit práci s rastrovými daty.
Joint Photographic Experts Group neboli JPEG komprese je komprese statických a i pohyblivých rastrů. Je vhodná pro kompresi přirozených rastrů a není vhodná u rastrů vzniklých z vektorových rastrů. JPEG komprese funguje na faktu že lidské oko vnímá barvu pomocí čípků a tyčinek, přičemž tyčinky jsou citlivé na jas a čípky na barvu. V lidském oku je přibližně 170 milionů tyčinek a jenom 6-7 milionů, z toho vyplývá že lidské oko je vnímavější ke změně jasu, než ke změně barvy. Nevýznamné změny barev jsou odstraňovány, změny jasu jsou naopak s co největší přesností uchovány\cite{pIDhmNtdwMgbcGoe}.\\
Jedním nejčastěji využívaných barevných modelů (RGB) pro nekomprimované rastrová data je založen na aditivním mícháním jednotlivých barevných složek (čím je větší sytost barevných složek, tím je výsledná barva světlejší). Model RGB však nevyhovuje pro JPEG kompresi model RGB je tak v prvním kroku převeden do modelu $YC_BC_R$. Popis barevných odstínů v tomto modelu je blízky skutečnému fyziologickému vnímání barev. Model je představován třemi složkami jednou jasovou $Y$ a dvěma chrominančními $C_B$ a $C_R$\cite{pIDhmNtdwMgbcGoe}.\\
JPEG komprese se skládá z několika kroků\cite{pIDhmNtdwMgbcGoe}:
\begin{itemize}
    \item Převod $(RGB) \longrightarrow (Y,C_B,C_R)$
    \item Downsamplování rastru
    \item Transformace pomocí diskrétní kosinové transformace
    \item Kvantizace pomocí kvantizační matice
    \item ZIG-ZAG sekvence
    \item Huffmanovo kódování
\end{itemize}
JPEG dekomprese se skládá z opačného postupu než při kompresi\cite{pIDhmNtdwMgbcGoe}:
\begin{itemize}
    \item Huffmanovo kódování dekomprese
    \item Převod ZIG-ZAG sekvence na matice
    \item Dekvantizace pomocí kvantizační matice
    \item Transformace pomocí inverzní diskrétní kosinové transformace
    \item Upnsamplování rastru
    \item Převod $(Y,C_B,C_R) \longrightarrow (RGB)$
\end{itemize}
