\section*{Postup}

\subsection*{Načtení a zobrazení obrázku}
Obrázek je načten pomocí funkce \texttt{imread} a zobrazen pomocí \texttt{imshow}:
\begin{verbatim}
originalImage = imread('colour_2.bmp');
figure(1)
imshow(originalImage);
title('Uncompressed Image');
\end{verbatim}

\subsection*{Volba parametrů komprese}
Jsou nastaveny parametry komprese, jako je faktor komprese \texttt{q}, typ transformace \texttt{type\_of\_trans}, typ resamplingu \texttt{type\_of\_resample} a zda se má použít Huffmanovo kódování \texttt{use\_huffman}:
\begin{verbatim}
q = 50;
type_of_trans = 'dwt';
type_of_resample = '2X2';
use_huffman = 'YES';
\end{verbatim}

\subsection*{Transformace RGB do YCBCR modelu}
RGB složky obrázku jsou převedeny do YCBCR modelu pomocí následujících rovnic\cite{pIDhmNtdwMgbcGoe}\cite{ygeiGeoinf2}:
\[
\begin{pmatrix}
Y \\
CB \\
CR
\end{pmatrix}
=
\begin{pmatrix}
0.2990 & 0.5870 & 0.1140 \\
-0.1687 & -0.3313 & 0.5000 \\
0.5000 & -0.4187 & -0.0813
\end{pmatrix}
\begin{pmatrix}
R \\
G \\
B
\end{pmatrix}
+
\begin{pmatrix}
0 \\
128 \\
128
\end{pmatrix}
\]

\subsection*{Downsampling chrominančních složek}
Chrominanční složky (Cb a Cr) jsou downsamplovány pomocí zvolené metody (2X2 nebo NN):
\begin{verbatim}
CB_downsampled = feval(strcat('Resample.MyDResample', type_of_resample), CB);
CR_downsampled = feval(strcat('Resample.MyDResample', type_of_resample), CR);
\end{verbatim}

\subsection*{Kvantizace koeficientů}
Jsou definovány kvantizační matice pro Y a C složky a aktualizovány podle faktoru komprese \texttt{q}\cite{pIDhmNtdwMgbcGoe}\cite{ygeiGeoinf2}:
\[
Qy = \frac{50}{q} \cdot \begin{pmatrix}
16 & 11 & 10 & 16 & 24 & 40 & 51 & 61 \\
12 & 12 & 14 & 19 & 26 & 58 & 60 & 55 \\
14 & 13 & 16 & 24 & 40 & 87 & 69 & 56 \\
14 & 17 & 22 & 29 & 51 & 87 & 80 & 62 \\
18 & 22 & 37 & 26 & 68 & 109 & 103 & 77 \\
24 & 35 & 55 & 64 & 81 & 104 & 113 & 92 \\
49 & 64 & 78 & 87 & 103 & 121 & 120 & 101 \\
72 & 92 & 95 & 98 & 112 & 100 & 103 & 99
\end{pmatrix}
\]
\[
Qc = \frac{50}{q} \cdot \begin{pmatrix}
17 & 18 & 24 & 47 & 66 & 99 & 99 & 99 \\
18 & 21 & 26 & 66 & 99 & 99 & 99 & 99 \\
24 & 26 & 56 & 99 & 99 & 99 & 99 & 99 \\
47 & 69 & 99 & 99 & 99 & 99 & 99 & 99 \\
99 & 99 & 99 & 99 & 99 & 99 & 99 & 99 \\
99 & 99 & 99 & 99 & 99 & 99 & 99 & 99 \\
99 & 99 & 99 & 99 & 99 & 99 & 99 & 99 \\
99 & 99 & 99 & 99 & 99 & 99 & 99 & 99
\end{pmatrix}
\]

\subsection*{JPEG komprese a dekomprese}
Provedení JPEG komprese a dekomprese zahrnuje transformaci, kvantizaci a případné Huffmanovo kódování:
\begin{verbatim}
[Y_zigzag] = compression(Y, Qy, type_of_trans);
[CB_zigzag] = compression(CB_downsampled, Qc, type_of_trans);
[CR_zigzag] = compression(CR_downsampled, Qc, type_of_trans);

if strcmp(use_huffman, 'YES')
    % Huffmanovo kódování
    % ...
end

[Y] = decompression(Y_zigzag, Qy, type_of_trans);
[Cb] = decompression(CB_zigzag, Qc, type_of_trans);
[Cr] = decompression(CR_zigzag, Qc, type_of_trans);
\end{verbatim}

\subsection*{Upsampling Chrominančích složek}
Chrominančí složky jsou upsamplovány zpět na původní velikost:
\begin{verbatim}
Cb = feval(strcat('Resample.MyUResample', type_of_resample), Cb);
Cr = feval(strcat('Resample.MyUResample', type_of_resample), Cr);
\end{verbatim}

\subsection*{Transformace zpět do RGB}
YCBCR složky jsou převedeny zpět do RGB formátu\cite{pIDhmNtdwMgbcGoe}\cite{ygeiGeoinf2}:
\[
\begin{pmatrix}
R \\
G \\
B
\end{pmatrix}
=
\begin{pmatrix}
1.0000 & 0.0000 & 1.4020 \\
1.0000 & -0.3441 & -0.7141 \\
1.0000 & 1.7720 & -0.0001
\end{pmatrix}
\begin{pmatrix}
Y \\
CB - 128 \\
CR - 128
\end{pmatrix}
\]

\begin{verbatim}
Rd = Y + 1.4020*(Cr-128);
Gd = Y - 0.3441*(Cb-128) - 0.7141*(Cr-128);
Bd = Y + 1.7720 * (Cb-128) - 0.0001*(Cr-128);

Ri = uint8(Rd);
Gi = uint8(Gd);
Bi = uint8(Bd);

ras2(:,:,1) = Ri;
ras2(:,:,2) = Gi;
ras2(:,:,3) = Bi;

figure(2)
imshow(ras2);
title('Compressed Image');
\end{verbatim}

\subsection*{Výpočet střední kvadratické odchylky}
Střední kvadratická odchylka jednotlivých RGB složek je vypočtena podle následujícího vzorce:
\[
m = \sqrt{\frac{\sum_{i=0}^{m \cdot n} (z_i - z'_i)^2}{m \cdot n}}
\]

V MATLABu je tento výpočet proveden následujícím způsobem:
\begin{verbatim}
dR = R - Rd;
dG = G - Gd;
dB = B - Bd;

dR2 = dR.^2;
dG2 = dG.^2;
dB2 = dB.^2;

sigR = sqrt(sum(sum(dR2))/(m*n));
sigG = sqrt(sum(sum(dG2))/(m*n));
sigB = sqrt(sum(sum(dB2))/(m*n));
\end{verbatim}

Kde:
- \( dR, dG, dB \) jsou rozdíly mezi původními a dekomprimovanými složkami.
- \( dR2, dG2, dB2 \) jsou druhé mocniny těchto rozdílů.
- \( sigR, sigG, sigB \) jsou výsledné střední kvadratické odchylky pro jednotlivé složky.

Tímto způsobem je možné kvantifikovat rozdíly mezi původním a dekomprimovaným obrazem pro každou složku RGB.

\subsubsection*{Náhrada DCT s využitím diskrétní Fourierovy transformace (DFT)}

Jako alternativa k diskrétní kosinové transformaci (DCT) byla implementována diskrétní Fourierova transformace (DFT). DFT byla implementováná pomocí algoritmu fft.\cite{FFTWiki}

\subsubsection*{Náhrada DCT s využitím diskrétní vlnkové transformace (DWT)}

Další alternativou k DCT diskrétní vlnková transformace (DWT). DWT je schopna lépe zachytit detaily v obraze a může být výhodnější pro kompresi obrazů s vysokým rozlišením nebo složitými texturami.\cite{WaveletWiki}