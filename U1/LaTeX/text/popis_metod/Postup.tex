\section{Postup}

\subsection{Načtení a zobrazení obrázku}
Obrázek je načten pomocí funkce \texttt{imread} a zobrazen pomocí \texttt{imshow}:
\begin{verbatim}
originalImage = imread('colour_2.bmp');
figure(1)
imshow(originalImage);
title('Uncompressed Image');
\end{verbatim}

\subsection{Volba parametrů komprese}
Jsou nastaveny parametry komprese, jako je faktor komprese \texttt{q}, typ transformace \texttt{type\_of\_trans}, typ resamplingu \texttt{type\_of\_resample} a zda se má použít Huffmanovo kódování \texttt{use\_huffman}:
\begin{verbatim}
q = 50;
type_of_trans = 'dwt';
type_of_resample = '2X2';
use_huffman = 'YES';
\end{verbatim}

\subsection{Transformace RGB do YCBCR modelu}
RGB složky obrázku jsou převedeny do YCBCR modelu pomocí následujících rovnic\cite{pIDhmNtdwMgbcGoe}\cite{ygeiGeoinf2}:
\[
\begin{pmatrix}
Y \\
CB \\
CR
\end{pmatrix}
=
\begin{pmatrix}
0.2990 & 0.5870 & 0.1140 \\
-0.1687 & -0.3313 & 0.5000 \\
0.5000 & -0.4187 & -0.0813
\end{pmatrix}
\begin{pmatrix}
R \\
G \\
B
\end{pmatrix}
+
\begin{pmatrix}
0 \\
128 \\
128
\end{pmatrix}
\]

\subsection{Downsampling chrominančních složek}
Chrominanční složky (Cb a Cr) jsou downsamplovány pomocí zvolené metody (2X2 nebo NN):
\begin{verbatim}
CB_downsampled = feval(strcat('Resample.MyDResample', type_of_resample), CB);
CR_downsampled = feval(strcat('Resample.MyDResample', type_of_resample), CR);
\end{verbatim}

\subsection{Kvantizace koeficientů}
Jsou definovány kvantizační matice pro Y a C složky a aktualizovány podle faktoru komprese \texttt{q}\cite{pIDhmNtdwMgbcGoe}\cite{ygeiGeoinf2}:
\[
Qy = \frac{50}{q} \cdot \begin{pmatrix}
16 & 11 & 10 & 16 & 24 & 40 & 51 & 61 \\
12 & 12 & 14 & 19 & 26 & 58 & 60 & 55 \\
14 & 13 & 16 & 24 & 40 & 87 & 69 & 56 \\
14 & 17 & 22 & 29 & 51 & 87 & 80 & 62 \\
18 & 22 & 37 & 26 & 68 & 109 & 103 & 77 \\
24 & 35 & 55 & 64 & 81 & 104 & 113 & 92 \\
49 & 64 & 78 & 87 & 103 & 121 & 120 & 101 \\
72 & 92 & 95 & 98 & 112 & 100 & 103 & 99
\end{pmatrix}
\]
\[
Qc = \frac{50}{q} \cdot \begin{pmatrix}
17 & 18 & 24 & 47 & 66 & 99 & 99 & 99 \\
18 & 21 & 26 & 66 & 99 & 99 & 99 & 99 \\
24 & 26 & 56 & 99 & 99 & 99 & 99 & 99 \\
47 & 69 & 99 & 99 & 99 & 99 & 99 & 99 \\
99 & 99 & 99 & 99 & 99 & 99 & 99 & 99 \\
99 & 99 & 99 & 99 & 99 & 99 & 99 & 99 \\
99 & 99 & 99 & 99 & 99 & 99 & 99 & 99 \\
99 & 99 & 99 & 99 & 99 & 99 & 99 & 99
\end{pmatrix}
\]

\subsection{JPEG komprese a dekomprese}
Provedení JPEG komprese a dekomprese zahrnuje transformaci, kvantizaci a případné Huffmanovo kódování:
\begin{verbatim}
[Y_zigzag] = compression(Y, Qy, type_of_trans);
[CB_zigzag] = compression(CB_downsampled, Qc, type_of_trans);
[CR_zigzag] = compression(CR_downsampled, Qc, type_of_trans);

if strcmp(use_huffman, 'YES')
    % Huffmanovo kódování
    % ...
end

[Y] = decompression(Y_zigzag, Qy, type_of_trans);
[Cb] = decompression(CB_zigzag, Qc, type_of_trans);
[Cr] = decompression(CR_zigzag, Qc, type_of_trans);
\end{verbatim}

\subsection{Upsampling Chrominančích složek}
Chrominanční složky jsou upsamplovány zpět na původní velikost:
\begin{verbatim}
Cb = feval(strcat('Resample.MyUResample', type_of_resample), Cb);
Cr = feval(strcat('Resample.MyUResample', type_of_resample), Cr);
\end{verbatim}

\subsection{Transformace zpět do RGB}
YCBCR složky jsou převedeny zpět do RGB formátu\cite{pIDhmNtdwMgbcGoe}\cite{ygeiGeoinf2}:
\[
\begin{pmatrix}
R \\
G \\
B
\end{pmatrix}
=
\begin{pmatrix}
1.0000 & 0.0000 & 1.4020 \\
1.0000 & -0.3441 & -0.7141 \\
1.0000 & 1.7720 & -0.0001
\end{pmatrix}
\begin{pmatrix}
Y \\
CB - 128 \\
CR - 128
\end{pmatrix}
\]

\begin{verbatim}
Rd = Y + 1.4020*(Cr-128);
Gd = Y - 0.3441*(Cb-128) - 0.7141*(Cr-128);
Bd = Y + 1.7720 * (Cb-128) - 0.0001*(Cr-128);

Ri = uint8(Rd);
Gi = uint8(Gd);
Bi = uint8(Bd);

ras2(:,:,1) = Ri;
ras2(:,:,2) = Gi;
ras2(:,:,3) = Bi;

figure(2)
imshow(ras2);
title('Compressed Image');
\end{verbatim}

\subsection{Výpočet střední kvadratické odchylky}
Střední kvadratická odchylka jednotlivých RGB složek je vypočtena podle následujícího vzorce:
\[
m = \sqrt{\frac{\sum_{i=0}^{m \cdot n} (z_i - z'_i)^2}{m \cdot n}}
\]

V MATLABu je tento výpočet proveden následujícím způsobem:
\begin{verbatim}
dR = R - Rd;
dG = G - Gd;
dB = B - Bd;

dR2 = dR.^2;
dG2 = dG.^2;
dB2 = dB.^2;

sigR = sqrt(sum(sum(dR2))/(m*n));
sigG = sqrt(sum(sum(dG2))/(m*n));
sigB = sqrt(sum(sum(dB2))/(m*n));
\end{verbatim}

Kde:
- \( dR, dG, dB \) jsou rozdíly mezi původními a dekomprimovanými složkami.
- \( dR2, dG2, dB2 \) jsou druhé mocniny těchto rozdílů.
- \( sigR, sigG, sigB \) jsou výsledné střední kvadratické odchylky pro jednotlivé složky.

Tímto způsobem je možné kvantifikovat rozdíly mezi původním a dekomprimovaným obrazem pro každou složku RGB.

\subsection{Náhrada DCT s využitím diskrétní Fourierovy transformace (DFT)}

Jako alternativa k diskrétní kosinové transformaci (DCT) byla implementována diskrétní Fourierova transformace (DFT). DFT byla implementováná pomocí algoritmu FFT.\cite{FFTWiki} Diskrétní Fourierova transformace (DFT) je definována jako:

\[
X(k) = \sum_{n=0}^{N-1} x(n) \cdot e^{-j \frac{2\pi}{N} kn}
\]
kde:
\begin{itemize}
    \item \( X(k) \) je k-tý koeficient ve frekvenční doméně,
    \item \( x(n) \) je n-tý vzorek v prostorové doméně,
    \item \( N \) je počet vzorků,
    \item \( j \) je imaginární jednotka \((j = \sqrt{-1})\).
\end{itemize}

\subsubsection{2D Fourierova transformace}
Pro zpracování obrazu (2D matice) se využívá 2D DFT, která je definována jako:
\[
X(u, v) = \sum_{x=0}^{M-1} \sum_{y=0}^{N-1} f(x, y) \cdot e^{-j 2\pi \left( \frac{ux}{M} + \frac{vy}{N} \right)}
\]
kde:
\begin{itemize}
    \item \( X(u, v) \) je koeficient ve frekvenční doméně,
    \item \( f(x, y) \) je hodnota pixelu v prostorové doméně,
    \item \( M \) a \( N \) jsou rozměry obrazu,
    \item \( j \) je imaginární jednotka (j = \(\sqrt{-1}\)).
\end{itemize}

\subsubsection{Rychlá Fourierova transformace (FFT)}
Rychlá Fourierova transformace (FFT) je optimalizovaný algoritmus pro výpočet DFT, který výrazně snižuje počet potřebných operací. FFT využívá rekurzi k rozdělení DFT na menší části. Matematicky je FFT definována jako:
\[
X(k) = \sum_{m=0}^{N/2-1} x(2m) \cdot e^{-j \frac{2\pi}{N/2} km} + e^{-j \frac{2\pi}{N} k} \sum_{m=0}^{N/2-1} x(2m+1) \cdot e^{-j \frac{2\pi}{N/2} km}
\]

\subsubsection{Použité proměnné}
\begin{itemize}
    \item \( X(k) \): Výsledek FFT pro k-tou frekvenci.
    \item \( x(n) \): Vstupní signál v časové (prostorové) doméně.
    \item \( N \): Počet vzorků (délka vstupního signálu).
    \item \( k \): Index frekvence, pro kterou se FFT počítá.
    \item \( m \): Index pro iteraci přes polovinu vzorků.
    \item \( j \): Imaginární jednotka \((j = \sqrt{-1})\).
\end{itemize}

\subsubsection*{Rekurze v FFT}
FFT algoritmus využívá rekurzi k rozdělení DFT na menší části. Tento přístup umožňuje efektivní výpočet transformace i pro velké množství dat. Rekurzivní algoritmus FFT je definován jako:
\begin{verbatim}
function F = fft(x)
    n = length(x);
    if n <= 1
        F = x;
    else
        even = fft(x(1:2:end));
        odd = fft(x(2:2:end));
        F = zeros(1, n);
        for k = 1:n/2
            t = exp(-2i * pi * (k-1) / n) * odd(k);
            F(k) = even(k) + t;
            F(k + n/2) = even(k) - t;
        end
    end
end
\end{verbatim}

\subsubsection{Dvourozměrná Fourierova transformace (FFT2)}
Funkce \texttt{fft2} provádí dvourozměrnou Fourierovu transformaci na matici. Nejprve se provede FFT na každém řádku matice a poté na každém sloupci výsledné matice.
\newpage
\begin{verbatim}
function F = fft2(X)
    [rows, cols] = size(X);
    F_complex = zeros(rows, cols);
    for i = 1:rows
        F_complex(i, :) = fft(X(i, :));
    end
    for i = 1:cols
        F_complex(:, i) = fft(F_complex(:, i));
    end
    F = F_complex;
end
\end{verbatim}

\subsubsection{Inverzní dvourozměrná Fourierova transformace (IFFT2)}
Funkce \texttt{ifft2} provádí inverzní dvourozměrnou Fourierovu transformaci na matici. Nejprve se provede IFFT na každém sloupci matice a poté na každém řádku výsledné matice.

\begin{verbatim}
function x = ifft2(X)
    [rows, cols] = size(X);
    x_complex = zeros(rows, cols);
    for c = 1:cols
        x_complex(:, c) = ifft(X(:, c).');
    end
    for r = 1:rows
        x_complex(r, :) = ifft(x_complex(r, :));
    end
    x = x_complex;
end
\end{verbatim}

\subsubsection{Inverzní Fourierova transformace (IFFT)}
Inverzní Fourierova transformace (IFFT) převádí obraz z frekvenční domény zpět do prostorové domény. Matematicky je definována jako:
\[
x(n) = \frac{1}{N} \sum_{k=0}^{N-1} X(k) \cdot e^{j \frac{2\pi}{N} kn}
\]
kde:
\begin{itemize}
    \item \( x(n) \) je n-tý vzorek v prostorové doméně,
    \item \( X(k) \) je k-tý koeficient ve frekvenční doméně,
    \item \( N \) je počet vzorků,
    \item \( j \) je imaginární jednotka \((j = \sqrt{-1})\).
\end{itemize}

Algoritmus IFFT je podobný FFT, ale s opačným znaménkem v exponentu:
\begin{verbatim}
function x = ifft(X)
    n = length(X);
    if n <= 1
        x = X;
    else
        even = ifft(X(1:2:end));
        odd = ifft(X(2:2:end));
        x = zeros(1, n);
        for k = 1:n/2
            t = exp(2i * pi * (k-1) / n) * odd(k);
            x(k) = (even(k) + t) / 2;
            x(k + n/2) = (even(k) - t) / 2;
        end
    end
end
\end{verbatim}


\subsection{Náhrada DCT s využitím diskrétní vlnkové transformace (DWT)}

Další alternativou k DCT je diskrétní vlnková transformace (DWT). DWT je schopna lépe zachytit detaily v obraze a může být výhodnější pro kompresi obrazů s vysokým rozlišením nebo složitými texturami.\cite{WaveletWiki} Matematicky je DWT definována jako:

\[
W_{\psi}[j, k] = \sum_{n=0}^{N-1} x[n] \psi_{j,k}[n]
\]

kde:
\begin{itemize}
    \item \(j\) je měřítko (scale),
    \item \(k\) je posun (translation),
    \item \(\psi_{j,k}[n]\) je diskrétní vlnková funkce.
\end{itemize}

\subsubsection{Použitý druh vlnky}
V této implementaci byla použita Haarova vlnka, která je poměrně jednoduchá. Haarova vlnka je definována jako:

\[
\psi(t) = \begin{cases} 
1 & \text{pro } 0 \leq t < \frac{1}{2} \\
-1 & \text{pro } \frac{1}{2} \leq t < 1 \\
0 & \text{jinak}
\end{cases}
\]

\[
\phi(t) = \begin{cases} 
1 & \text{pro } 0 \leq t < 1 \\
0 & \text{jinak}
\end{cases}
\]

\subsubsection{Dvourozměrná diskrétní vlnková transformace (DWT2)}
Pro zpracování obrazu se využívá DWT2, která se provádí aplikací DWT na řádky a následně na sloupce obrazu. Výsledkem jsou čtyři sub-pásma: LL (nízkopásmové), LH (nízkopásmové horizontální), HL (nízkopásmové vertikální) a HH (vysokopásmové). Algoritmus 2D DWT pro zpracování obrazu je následující:
\begin{verbatim}
function output = dwt2(block)
    [rows, cols] = size(block);
    L = zeros(rows, cols/2);
    H = zeros(rows, cols/2);
    for i = 1:rows
        even_cols = block(i, 1:2:end);
        odd_cols = block(i, 2:2:end);
        L(i, :) = (even_cols + odd_cols) / 2;
        H(i, :) = (even_cols - odd_cols) / 2;
    end
    LL = zeros(rows/2, cols/2);
    LH = zeros(rows/2, cols/2);
    HL = zeros(rows/2, cols/2);
    HH = zeros(rows/2, cols/2);
    for j = 1:cols/2
        even_rows_L = L(1:2:end, j);
        odd_rows_L = L(2:2:end, j);
        even_rows_H = H(1:2:end, j);
        odd_rows_H = H(2:2:end, j);
        LL(:, j) = (even_rows_L + odd_rows_L) / 2;
        LH(:, j) = (even_rows_L - odd_rows_L) / 2;
        HL(:, j) = (even_rows_H + odd_rows_H) / 2;
        HH(:, j) = (even_rows_H - odd_rows_H) / 2;
    end
    output = [LL, LH; HL, HH];
end
\end{verbatim}

\subsubsection{Inverzní 2D DWT (IDWT2)}
Inverzní diskrétní vlnková transformace (IDWT) obnovuje původní signál z jeho vlnkových koeficientů. Matematicky je IDWT definována jako:

\[
x[n] = \sum_{j,k} W_{\psi}[j, k] \psi_{j,k}[n]
\]

kde:
\begin{itemize}
    \item \(W_{\psi}[j, k]\) jsou vlnkové koeficienty,
    \item \(\psi_{j,k}[n]\) jsou diskrétní vlnkové funkce.
\end{itemize}
Algoritmus 2D IDWT pro zpracování obrazu je následující:
\begin{verbatim}
function block = idwt2([LL, LH; HL, HH])
    [rows, cols] = size(LL);
    LL_sub = LL(1:4, 1:4);
    LH_sub = LL(1:4, 5:8);
    HL_sub = LL(5:8, 1:4);
    HH_sub = LL(5:8, 5:8);
    L = zeros(rows, cols/2);
    H = zeros(rows, cols/2);
    for j = 1:cols/2
        even_rows_L = LL_sub(:, j) + LH_sub(:, j);
        odd_rows_L = LL_sub(:, j) - LH_sub(:, j);
        even_rows_H = HL_sub(:, j) + HH_sub(:, j);
        odd_rows_H = HL_sub(:, j) - HH_sub(:, j);
        L(1:2:end, j) = even_rows_L;
        L(2:2:end, j) = odd_rows_L;
        H(1:2:end, j) = even_rows_H;
        H(2:2:end, j) = odd_rows_H;
    end
    block = zeros(rows, cols);
    for i = 1:rows
        even_cols = L(i, :) + H(i, :);
        odd_cols = L(i, :) - H(i, :);
        block(i, 1:2:end) = even_cols;
        block(i, 2:2:end) = odd_cols;
    end
end
\end{verbatim}