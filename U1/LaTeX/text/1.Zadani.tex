\section{Zadání:}

Implementujte algoritmus pro JPEG kompresi/dekompresi rastru v prostředí MATLAB (popř. v programovacím jazyce dle vlastního výběru), zahrnující tyto fáze:
\begin{itemize}
    \item transformaci do $YC_BC_R$ modelu,
    \item diskrétní kosinovou transformaci,
    \item kvantizaci koeficientů,
\end{itemize}
a to \textbf{bez} využití vestavěných funkcí

Kompresní algoritmus otestujte na různých typech rastru: rastr v odstínech šedi, barevný rastr (viz tabulka) vhodného rozlišení a velikosti (max 128x128 pixelů) s různými hodnotami faktoru komprese $q = 10,50,70$\\
Pro každou variantu spočtěte střední kvadratickou odchylku $m$ jednotlivých RGB složek:
$$m=\sqrt{\frac{\sum_{i=0}^{mn}(z-z')^2}{mn}}$$
Výsledky umístěte do přehledných tabulek pro jednotlivá q. Na základě výše vypočtených údajů zhodnoťte, ke kterým typům dat je JPEG komprese nejvíce a naopak nejméně vhodná.