\section{Popis metod}
Pro výpočet shlukování byla vytvořena třída \texttt{Clustering}, která obsahuje metody pro různé algoritmy shlukování: k-means (\texttt{kmeans}), hierarchické shlukování (\texttt{hierar}) a DBSCAN (\texttt{dbscan}).
\subsection{K-means}

K-means přiřazuje body k nejbližším centroidům, probíhá v několika iteracích, kdy se na základě aktuálních poloh centroidů přiřazují body k jednotlivým shlukům a následně se centroidy přepočítávají. Tento proces pokračuje, dokud se pozice centroidů stabilizují nebo dokud není dosaženo maximálního počtu iterací. Popis výpočtu k-means je uveden v následujícím pseudokódu:

\begin{algorithm}
    \setstretch{1.2}
    \caption{Metoda \texttt{kmeans}}
    \begin{algorithmic}[1]
        \STATE \textbf{Vstup:} $M$ -- Matice bodů, $k$ -- Počet klastrů, $max\_iter$ -- Počet iterací, $PS$ -- Tolerance konvergence\\
        \STATE \textbf{Výstup:} $S$ -- Pozice centroidů, $L$ -- Přiřazení bodů ke klastrům
        \STATE $Max, Min \gets \text{max}(M), \text{min}(M)$ // Určení rozsahu dat
        \STATE $S \gets Min + (Max - Min) \cdot \text{rand}(k, \text{size}(M, 2))$ // Inicializace centroidů
        \STATE $N \gets 0$, $N\_max \gets max\_iter$
        
        \WHILE{$N < N\_max$}
            \STATE $D \gets \text{pdist2}(M, S)$ // Výpočet vzdáleností
            \STATE $L \gets \text{argmin}(D, \text{axis}=2)$ // Přiřazení bodů
            \STATE $S\_new \gets \text{zeros}(k, \text{size}(M, 2))$
            \FOR{$j \gets 1$ \TO $k$}
                \STATE $cluster\_points \gets M[L == j, :]$
                \IF{$\text{isempty}(cluster\_points)$}
                    \STATE $S\_new[j, :] \gets Min + (Max - Min) \cdot \text{rand}(1, \text{size}(M, 2))$ // Inicializace nového centroidu
                \ELSE
                    \STATE $S\_new[j, :] \gets \text{mean}(cluster\_points, \text{axis}=1)$ // Výpočet nového centroidu
                \ENDIF
            \ENDFOR
            \STATE $diff \gets \text{norm}(S\_new - S)$ // Kontrola konvergence
            \IF{$diff < PS$} \STATE \textbf{break} \ENDIF
            \STATE $S \gets S\_new$, $N \gets N + 1$
        \ENDWHILE
        \STATE \textbf{return} $S, L$
    \end{algorithmic}
\end{algorithm}
\newpage

\subsection{Hierarchické shlukování}

Hierarchické shlukování je metoda, která vytváří hierarchii shluků tím, že postupně spojuje (aglomerační přístup) data na základě vzdálenosti mezi jednotlivými body. Popis výpočtu hierarchického shlukování je uveden v následujícím pseudokódu:

\begin{algorithm}
    \setstretch{1.2}
    \caption{Metoda \texttt{hierar}}
    \begin{algorithmic}[1]
        \STATE \textbf{Vstup:} $M$ -- Matice bodů, $k$ -- Počet klastrů\\
        \STATE \textbf{Výstup:} $clusters$ -- Seznam shluků, kde každý shluk obsahuje indexy bodů v daném shluku
        \STATE $distance\_matrix \gets \text{pdist2}(M, M)$ // Výpočet matice vzdáleností
        \STATE $n \gets \text{size}(M, 1)$ // Počet bodů
        \STATE $clusters \gets \text{num2cell}(1:n)$ // Inicializace klastrů, každý bod je ve svém vlastním shluku
        
        \FOR{$i \gets 1$ \TO $n + 1 - k$}
            \STATE $min\_val \gets \min(distance\_matrix(:))$ // Minimalní hodnota v matici
            \STATE $[row\_indices, col\_indices] \gets \text{find}(distance\_matrix == min\_val)$ // Indexy min hodnot
            \STATE $valid\_pairs \gets (row\_indices \neq col\_indices)$ // Filtrace párů, kde indexy jsou různé
            \STATE $row\_indices \gets row\_indices(valid\_pairs)$, $col\_indices \gets col\_indices(valid\_pairs)$
            
            \FOR{$j \gets 1$ \TO \text{length}(row\_indices)}
                \STATE $row \gets row\_indices(j)$, $col \gets col\_indices(j)$
                \STATE $clusters[row] \gets clusters[row] \cup clusters[col]$ // Spojení shluků
                \STATE $clusters[col] \gets []$ // Vyprázdnění sloučeného shluku
            \ENDFOR

            \STATE $distance\_matrix \gets \text{inf}(\text{length}(clusters))$ // Resetování matice
            \FOR{$m \gets 1$ \TO \text{length}(clusters)}
                \FOR{$n \gets 1$ \TO \text{length}(clusters)}
                    \IF{$m \neq n$ \textbf{and} $clusters[m] \neq []$ \textbf{and} $clusters[n] \neq []$}
                        \STATE $dist \gets \min(\text{pdist2}(M(clusters[m], :), M(clusters[n], :)), [], \text{all})$ 
                        \STATE $distance\_matrix[m, n] \gets dist$
                        \STATE $distance\_matrix[n, m] \gets dist$
                    \ENDIF
                \ENDFOR
            \ENDFOR
        \ENDFOR

        \STATE $empty\_indices \gets \textit{cellfun}(\texttt{'isempty'}, clusters)$ // Hledání prázdných shluků
        \STATE $clusters \gets clusters(\neg empty\_indices)$ // Výběr neprázdných shluků
        \STATE \textbf{return} $clusters$
    \end{algorithmic}
\end{algorithm}

\subsection{DBSCAN}

DBSCAN (Density-Based Spatial Clustering of Applications with Noise) se zaměřuje na identifikaci hustých oblastí v prostoru a ignoruje šum (tj. body, které nepatří k žádnému shluku). Tento algoritmus nevyžaduje stanovení počtu shluků předem. Popis výpočtu DBSCAN je uveden v následujícím pseudokódu:

\begin{algorithm}
    \setstretch{1.2}
    \caption{Metoda \texttt{dbscan}}
    \begin{algorithmic}[1]
        \STATE \textbf{Vstup:} $M$ -- Matice bodů, $epsilon$ -- Maximální vzdálenost mezi dvěma body pro označení sousedů, $minPts$ -- Minimální počet bodů pro označení husté oblasti (jádro)\\
        \STATE \textbf{Výstup:} $clusters$ -- Seznam shluků, kde každý shluk obsahuje indexy bodů v daném shluku
        \STATE $n \gets \text{size}(M, 1)$ // Počet bodů
        \STATE $labels \gets \text{zeros}(n, 1)$ // Inicializace popisků klastrů
        \STATE $cluster\_id \gets 0$ // Počáteční ID pro shluky
        \STATE $clusters \gets \text{cell}(n, 1)$ // Inicializace seznamu pro shluky

        \FOR{$i \gets 1$ \TO $n$}
            \IF{$labels[i] \neq 0$}
                \STATE \textbf{continue} // Pokud je bod již navštívený, přeskoč ho
            \ENDIF

            \STATE $neighbors \gets \text{find}(\text{pdist2}(M(i, :), M) \leq $epsilon$)$ // Hledání sousedů bodu
            \IF{$\text{numel}(neighbors) < minPts$}
                \STATE $labels[i] \gets -1$ // Označení bodu jako šumu
                \STATE \textbf{continue}
            \ENDIF

            \STATE $cluster\_id \gets cluster\_id + 1$ // Vytvoření nového shluku
            \STATE $labels[i] \gets cluster\_id$ // Přiřazení bodu k novému shluku
            \STATE $current\_cluster \gets neighbors$ // Aktuální seznam sousedních bodů

            \STATE $k \gets 1$
            \WHILE{$k \leq \text{numel}(current\_cluster)$}
                \STATE $j \gets current\_cluster(k)$
                \IF{$labels[j] = -1$}
                    \STATE $labels[j] \gets cluster\_id$ // Změna šumu na okrajový bod
                \ENDIF

                \IF{$labels[j] = 0$}
                    \STATE $labels[j] \gets cluster\_id$ // Přiřazení bodu k aktuálnímu shluku
                    \STATE $new\_neighbors \gets \text{find}(\text{pdist2}(M(j, :), M) \leq $epsilon$)$

                    \IF{$\text{numel}(new\_neighbors) \geq minPts$}
                        \STATE $current\_cluster \gets \text{union}(current\_cluster, new\_neighbors)$ // Nový sousedé do seznamu
                    \ENDIF
                \ENDIF

                \STATE $k \gets k + 1$
            \ENDWHILE

            \STATE $clusters[cluster\_id] \gets current\_cluster$ // Uložení bodů do aktuálního shluku
        \ENDFOR

        \STATE $empty\_indices \gets \textit{cellfun}(\texttt{'isempty'}, clusters)$ // Hledání prázdných shluků
        \STATE $clusters \gets clusters(\neg empty\_indices)$ // Výběr neprázdných shluků
        \STATE \textbf{return} $clusters$
    \end{algorithmic}
\end{algorithm}