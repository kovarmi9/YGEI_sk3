\section{Popis problému}

Shlukování je metoda neřízené klasifikace, která se používá k identifikaci podobných skupin dat. Cílem je rozdělit data do shluků, přičemž data v každém shluku jsou si co nejvíce podobná a shluky navzájem jsou co nejvíce odlišné. Shlukování se používá například v DPZ pro neříznou klasifikaci satelitních snímků.\cite{potuckova2024}

\subsection{Dílčí kroky shlukové analýzy}

Shlukování zahrnuje několik klíčových kroků. Nejprve je třeba zjistit, zda jsou data vhodná pro shlukovou analýzu. Pokud data nemají tendenci vytvářet shluky v daném příznakovém prostoru, nebude shluková analýza účinná. Z tohoto důvodu je důležité provést následující kroky:

\begin{itemize}
    \item \textbf{Výběr příznaků}: Je nutné vybrat relevantní příznaky a minimalizovat redundanci nebo korelaci mezi nimi.
    \item \textbf{Míry podobnosti/rozdílnosti}: Určení, jak měřit "blízkost" nebo "rozdílnost" mezi jednotlivými body.
    \item \textbf{Rozhodovací kritérium}: Výběr metody pro optimalizaci výsledků, např. pomocí nákladové funkce.
    \item \textbf{Výběr algoritmu shlukování}: Volí se na základě charakteristik dat.
    \item \textbf{Validace}: Ověření kvality výsledků.
    \item \textbf{Interpretace výsledků}: Posledním krokem je interpretace výsledků získaných ze shlukování.
\end{itemize}

\subsection{Přístupy ke shlukové analýze}

\begin{itemize}
    \item \textbf{Sekvenční přístup}: Tento přístup nevyžaduje předem stanovený počet shluků a algoritmus postupně přiřazuje data k shlukům na základě definovaných parametrů, jako je prahová hodnota vzdálenosti.
    \item \textbf{Hierarchický přístup}: Algoritmus spojuje (Aglomerativní hierarchické algoritmy) nebo dělí (Dělící hierarchické algoritmy) shluky na základě podobnosti mezi body. Vzniká struktura podobná binárnímu strmu.
    \item \textbf{Optimalizace nákladové funkce}: Založen na optimalizaci nákladové funkce \(J\) (funkce vektorů datové sady \(X\)) parametrizované neznámým vektorem \(\Theta\).
\end{itemize}

\subsection{Běžně používané algoritmy}

\begin{itemize}
    \item \textbf{K-means}: Algoritmus, který přiřazuje data k centroidům shluků, které se v každé iteraci přepočítávají. Je citlivý na šum v datech a počáteční inicializaci centroidů.
    \item \textbf{Hierarchické shlukování}: Tento přístup postupně spojuje nebo dělí shluky na základě jejich podobnosti. Je vhodný pro analýzu hierarchických vztahů mezi daty. Výsledky lze vizualizovat jako dendrogram, což umožňuje snadno pochopit strukturu dat.
    \item \textbf{DBSCAN}: Algoritmus identifikuje husté oblasti jako shluky a řídké oblasti ignoruje šum. Vstupní parametry, jako hustota bodů a okolí \(\varepsilon\), určují, které body patří do shluků, a to bez nutnosti předem definovat počet shluků.
    \item \textbf{Fuzzy shlukování}: Tento algoritmus umožňuje, aby každý bod patřil do více shluků s různou mírou příslušnosti, což je užitečné pro data, kde jsou hranice mezi shluky nejednoznačné nebo rozmazané.\cite{koutroumbas2008}
\end{itemize}

